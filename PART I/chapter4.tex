\chapter*{Conclusion générale}
\paragraph{}
En conclusion de ce travail, nous pouvons dire malgré la simplicité apparente d'un problème, il est très souvent impossible de le résoudre à l'aide de méthodes dites \textbf{classiques}, il est vrai qu'un taux de réussite de 97\% par exemple peut parraître suffisaint, on ne doit pas oublié que ce taux évolue selon la taille du problème, en effet sur les instances de tailels moyenne vue dans cette partie du tp, il aurait été préférable de trouver des méthodes qui avoisinent les 99\% de taux de réussite, mais il est évident que ces méthodes représentent les limites des méthodes classiques, c'est ainsi de façon naturelle et sensée, que nous allons passé des méthodes heuristiques aux méta-heuristiques, une évolution nécessaire pour ne serait ce qu'approximer de façon plausibles et suffisante la solution optimale cachée dérrière cet océan de solutions.